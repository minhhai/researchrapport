\documentclass[a4paper,10pt]{article}
\usepackage[utf8]{inputenc}


\usepackage[backend=bibtex,
style=numeric,
bibencoding=ascii
%style=alphabetic
%style=reading
]{biblatex}
\addbibresource{sample.bib}

\title{Research and information retriever: Crowd-based/crowdsourcing testing application}
\author{Pham Minh Hai}

\begin{document}

\maketitle
\tableofcontents

\section{Crowdbased Software}
 Crowdscourcing model  will be the future of software engineering. This concept has gained much attention over past few years, appstore of apple is one of best examples that we can use to explain what is Crowdsourcing sofware. 
 It is an open call for participation in any process or task of software development, include documentation, design, coding and testing. According Jeff Howe's definition : "\textit{ Crowdsourcing represent the act of a company or institution taking a function once performed by employees and outsourcing it to an undefined (adn generally large) network of people in the form of an open call."} \cite{crowdsourcing}
 
\section{Crowdbased-testing software}

Testing is one of the most important task in Software development. 
\textbf{Crowd-based testing}:\cite{Mobiletest} 
The crowd-based testing approach involves using freelance or contracted testing engineers or a community of end users such as uTest(www.utest.com). This Model of testing can be understood as ad hoc or managed testing. Generally, This testing approach have benefit of multinational collabroation of large and adaptive system for example more than 100,000 active tester freelances on uTest. However, this crowdbased testing has some challenges:\\
\begin{itemize}
	\item  Test operations are managed in an ad hoc way with very limited mobile test automation tools
	\item  This model can involves diversity of tester can lead to risk of low quality and uncertain vadidation schedule.
	\item 
	
\end{itemize}

\medskip
\printbibliography

\end{document}
